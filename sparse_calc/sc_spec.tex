\documentclass{article}

\begin{document}

\title{Sparse Matrix Calculator}

\subtitle{specification document}

\author{Peter Mills}

\maketitle

\section{Introduction}

Sparse matrices are invaluable for calculations relevant to many types of both physical and mathematical 
problemns. They are particularly useful for solving partial differential equations.
In the past, it was common when solving a sparse matrix to represent it directly
as code so that, for instance, a Fortran subroutine would perform a 
matrix multiplication with the specific type of matrix being solved.
This was common for instance not only in highly symmetric problems such as 
solving the Laplace or Helmholtz equations, but also when computing singular 
vectors of linearized prognostic weather simulations for use in ensemble forecasting and data assimilation.
In the latter case, the model itself serves as a stand in for the sparse Jacobi
matrix.

In the first case, such an approach has the advantage of speed while in the
second, simplicity. Both require less storage.
In either case, however, what has been lost is generality as a new program
must be written for any new type of problem.
Moreover, the storage requirements are not all that onerous since the matrices
themselves are sparse and typically scale linearly with the size of the problem
rather than to the square.

If we can adapt the physical simulations to generate sparse matrices
as output rather than integrating the computations into the models, 
a simple scripting language that can operate on sparse matrices using a small set
of operators could be extremely powerful.
Different types of operations and numerical experiments could be tried with
very little programming effort.

\section{General operation}

Like many computer languages, the software can be operated in two modes:
batch and interactive. Since variables correspond to files, source files can
be treated as either a main routine accepting arguments from the command
line or as a subroutine in another program.

Command line options can be used to specify both the location of the
data and of the source code for all the subroutines, e.g.:
\begin{verbatim}
sparse_calc [-p datapath] [-s path] [-h] [-e script] [-f source] [arg1 [arg2 [arg3 ...]]]
\end{verbatim}
where \verb\datapath\ is the data path where the variables are stored,
\verb\path\ is the path to the source files, and
\verb\-h\ prints the help screen.
Default path for both source and data is the current one.
\verb\source\ is the name of a file containing a program or subroutine
and \verb\arg1\, \verb\arg2\, etc are arguments to that program.
The \verb\-e\ option supplies a script that the software executes before
starting up in interactive mode.

\section{Types}

The language would have seven different types: literals, scalars, vectors, 
full matrices, sparse matrices, sparse matrix arrays, and lists.
We describe each in turn.

\subsection{Literals}

A literal is like a string except it is immutable. It has two main purposes:
to sever as a form of indirection similar to variable or function pointers and to pass options
to a function. For the first use, built-in operators will be supplied to take
the name of a subroutine and call that subroutine or to take the name of a variable
and perform operations, other than numerical, on that variable.
More on this later.
A literal is distinguished from a variable or subroutine by being enclosed
in double quotation marks.

\subsection{Scalars}

Scalars are double-precision floating point numerals.
They are represented in the usual fashion through one or more decimal
digits, plus optional decimal point, plus optional decimal fraction.

\subsection{Vectors}

Vectors are ordered lists of scalars. 
To create a vector of consecutive integers, use the \verb\:\ operator, e.g.:
\begin{verbatim}
> print(1:5)
1
2
3
4
5
\end{verbatim}
returns a vector containing the values $[1,2,3,4,5]$.
To create an empty vector (all zeroes) use the built-in function, 
\verb\empty\:
\begin{verbatim}
> print(empty(5))
0
0
0
0
0
\end{verbatim}
returns a vector containing the values, $[0, 0, 0 ,0,0]$.
To access an element of a vector, use the subscript operator, \verb\[]\:
\begin{verbatim}
> print(1:5[2])
3
\end{verbatim}
Note that subscripting is zero-based.

Storages pattern for vector data: straight binary dump of data [include type
info?]

\subsection{Full matrices}

A full matrix is a two-dimensional array of scalars. All values are stored,
including non-zero ones. This is what distinguishes them from sparse
matrices: see below.
For calculation, a full matrix should be indistinguishable
from a sparse matrix.
Care should be made, however, to choose the right type since calculations may
be slower for one type or the other depending on the properties of the matrix
in question.

Storage patter for matrices: four byte header giving number of columns for each
row plus straight binary dump of data, row-major.

\subsection{Sparse matrices}

A sparse matrix stores only the non-zero values of a matrix.
To create an empty sparse matrix, use the \verb\empty\ built-in:
\begin{verbatim}
> print(empty(3,3))
3 3
\end{verbatim}
To create an identity matrix, use the \verb\identity\ built-in:
\begin{verbatim}
> print(identity(5,5))
5 5
0 0 1
1 1 1
2 2 1
3 3 1
4 4 1
\end{verbatim}
This example nicely illustrates the storage scheme of a sparse matrix:
an orded pair of integers for the location of each non-zero element,
plus its value.

Like with vectors, elements of a matrix are accessed through subscripting.
A single subscript returns a row:
\begin{verbatim}
> print(identity(4,4)[2]
0
0
1
0
\end{verbatim}
while a double subscript returns a single elements.
Both forms, \verb\[i,j]\, and \verb\[i][j]\ are accepted, however note that
the second form is a compound operation returning first the row and then
subscripting that row. Therefore it is less efficient.

Storage pattern for sparse matrices: twelve-byte header comprising three
four-byte integers: number of rows, number of columns, number of non-zero
elements.
Each element is written in turn: row, column, value. Row and column are both
four-byte integers. Value can be either single or double-precision (eight
or sixteen bytes). Elements are sorted by location: row first, then column.

\subsection{Arrays of sparse matrices}

Sparse matrix arrays are ordered lists of sparse matrices which are treated
as a single matrix. To this end, when used in a calculation they are 
consecutively multiplied and they must be square and all the same size.
For matrices of different sizes, use the list type.
To extract a single matrix from the array, round brackets, \verb\()\, are 
used. Otherwise, the a sparse array can be treated as any other matrix
except that individual elements cannot be modified directly.

Storage: each sparse matrix is written consecutively in same format as above.

\subsection{List}

A list is an ordered collection of other numerical types, including other
lists. Like sparse arrays, when used in numerical calculations, they are
treated as a single object and multiplied through.
Like with sparse arrays, round brackets are used to access a single element 
of the list.

Storage: to be determined...

\section{Variables}

Each numeric variable correspond to a file with the exception of scalar
variables. The name of the variable is the same as the name of the file.
If a variable is declared within a session this means that it is a pre-existing
file and will not be deleted on exit.
If a variable is created (as opposed to declared) then the corresponding file
will be deleted at the end of a session.

Variables are declared using the following syntax:
\begin{verbatim}
> <type>(<name>)
\end{verbatim}
where <type> is one of: \verb\vector\, \verb\full\, \verb\sparse\
or \verb\sparse_array\ corresponding to the vector type, full matrix type
sparse matrix type and sparse matrix array type
and <name> is the name of a file containing data for the desired type.
[Note: issue with prototype; type system is poor: has all the disadvantages
of a static type system with none of the advantages...
would like to move to weak typing but this requires putting type information
into the files]

Variables are created through the assignment operator:
\begin{verbatim}
> a=1.5
\end{verbatim}
assigns the scalar value \verb\1.5\ to the variable, \verb\a\.
If \verb\a\ doesn't exist, it is created.

\subsection{Scoping}

Variables are scoped within a single source file which defines a single
routine/subroutine. More on this in the section on subroutines.

\section{Statements}

