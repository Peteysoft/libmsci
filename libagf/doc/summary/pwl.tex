\documentclass{article}

\newcommand{\decision}{f}
\newcommand{\testpoint}{\vec x}
\newcommand{\domainfunction}{\eta}
\newcommand{\normal}{v}
\newcommand{\constant}{b}
\newcommand{\dimension}{N}
\newcommand{\ndomain}{m}
\newcommand{\sample}{x}
\newcommand{\ord}{y}
\newcommand{\ntrain}{n}
\newcommand{\normals}{V}

\begin{document}

\title{Piecewise-linear statistical classifiers}

\author{Peter Mills}

\maketitle

\section{Introduction}

A piecewise linear classifier takes the following form:
\begin{equation}
\decision(\testpoint) = \vec \normal_{\domainfunction(\testpoint)} \cdot \testpoint + \constant_{\domainfunction(\testpoint)}
\end{equation}
where:
\begin{itemize}
  \item $\decision:\Re^\dimension \rightarrow \Re$ is the decision function which returns the class through its sign
  \item $\testpoint \in \Re^\dimension$ is a test point in a 
	  $\dimension$-dimensional feature space
  \item $\lbrace \vec \normal_i|~ i \in [1,\ndomain] \rbrace$ is a set of $\ndomain$
	  normals to the decision hyper-plane of each of the 
	  component linear classifiers,
  \item $\lbrace \constant_i|~ i \in [1,\ndomain] \rbrace$ are their corresponding intercepts
  \item $\domainfunction:\Re^\dimension \rightarrow [1:\ndomain]$ is an integer
function returning which component linear classifier to use for a given
test point.
\end{itemize}

While it looks simple, training this type of binary, 
statistical classifier, whether supervised or unsupervised, is not trivial.
Accurate methods tend to be slow while fast methods tend not to be accurate.
This paper reviews methods of constructing such classifiers from a set of 
training data: $\lbrace \vec \sample_i : \ord_i|~ i \in [1, \ntrain] \rbrace$
where $\ord_i \in [-1,~ 1]$.

Typically, the component linear classifier to use depends on some or all of
the others, thus:
\begin{equation}
	\domainfunction(\testpoint) = \domainfunction(\normals \testpoint + \vec \constant)
\end{equation}
where $\normals$ is a matrix combining all the normal vectors,
$\normals=[ \vec \normal_1, \vec \normal_2, \vec \normal_3, ... ]$ and 
$\vec \constant$ is a vector combining all the intercepts,
$\vec \constant= [ \constant_1, \constant_2, \constant_3,... ]$.

\end{document}
