\section{Sub-sampling}
\label{shuttle_subsampling}

Let $\classsize_i$ be the number of samples of the $i$th class such
that:
\begin{equation}
\classsize_i \ge \classsize_{i-1}
\end{equation}
Let $0 \le \subsample(n) \le 1$ be a function used to sub-sample each of the class
distributions in turn:
\begin{equation}
\classsize_i^\prime = \subsample(\classsize_i) \classsize_i
\end{equation}
We wish to retain the rank ordering of the class sizes:
\begin{equation}
\subsample(\classsize_i) \classsize_i 
\ge \subsample(\classsize_{i-1}) \classsize_{i-1} 
\end{equation}
while ensuring that the smallest classes have some minimum representation:
\begin{equation}
\subsample(\classsize_i) \le \subsample(\classsize_{i-1})
\label{subsample_constraint1}
\end{equation}
Thus:
\begin{equation}
	\frac{\mathrm d}{\mathrm d n} \left [ n \subsample(n) \right ] = \subsample(n) + n \frac{\mathrm d \subsample}{\mathrm n} \ge 0
\end{equation}
\begin{equation}
	\frac{\mathrm d \subsample}{\mathrm d n} \ge - \frac{\subsample(n)}{n}
\label{subsample_constraint2}
\end{equation}
The simplest means of ensuring that both (\ref{subsample_constraint1}) and
(\ref{subsample_constraint2}) are fulfilled, is to multiply the right side
of (\ref{subsample_constraint2}) with a constant, $0 \le \subexp \le 1$,
and equate it with the left side:
\begin{equation}
	\frac{\mathrm d \subsample}{\mathrm d n} = - \frac{\subexp \subsample(n)}{n}
\end{equation}
Integrating:
\begin{equation}
	\subsample(n)=\submultcoef n^{-\subexp}
\end{equation}
The parameter, $\subexp$, is set such that:
\begin{equation}
	\sum_i \submultcoef(n_i) n_i = \datafraction \sum_i n_i
\end{equation}
where $0 < \datafraction < 1$ is the desired fraction of training data.
With rearrangement:
\begin{equation}
	\frac{\submultcoef}{\datafraction} \sum_i n_i^\subexp = 0
\end{equation}

