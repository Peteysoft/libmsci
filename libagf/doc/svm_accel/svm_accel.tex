\documentclass[11pt]{article}

\bibliographystyle{apa}
\usepackage{natbib}
\usepackage[dvips]{graphicx}

\title{When Support Vector Machines are just too slow}

\begin{document}

%symbols for general kernel methods:
\newcommand{\vectorkernel}{K}
\newcommand{\scalarkernel}{K}
\newcommand{\kernelparam}{k}
\newcommand{\norm}{N}
\newcommand{\nsample}{n}
\newcommand{\dimension}{D}
\newcommand{\densityestimator}{\tilde p}
\newcommand{\bandwidth}{\sigma}
\newcommand{\coord}{x}
\newcommand{\samplecoord}{x}
\newcommand{\sample}{\vec \samplecoord}
\newcommand{\testcoord}{x}
\newcommand{\testpoint}{\vec \testcoord}
\newcommand{\point}{\vec \coord}
\newcommand{\classlabel}{y}
\newcommand{\class}{c}
\newcommand{\probability}{P}
\newcommand{\condprob}{p}
\newcommand{\kernelaverage}{W}

%symbols for SVM kernel method:
\newcommand{\svmcoeff}{w}
\newcommand{\expandedspace}{\vec\phi}
\newcommand{\svmdecision}{f}
\newcommand{\svmbordervector}{\vec v}
\newcommand{\svmborderconst}{b}

%borders classification
\newcommand{\decisionfunction}{\tilde r}
\newcommand{\diffcondprob}{r}
\newcommand{\kerneldecision}{\decisionfunction_{kern}}
\newcommand{\vbdecision}{\decisionfunction_{vb}}
\newcommand{\bordervector}{\vec b}
\newcommand{\bordernormal}{\vec v}
\newcommand{\nborder}{n_b}
\newcommand{\borderdecision}{g}
%derivative of variable bandwidth kernel estimator:
\newcommand{\distance}{d}
\newcommand{\scalarkernelderiv}{\dot \scalarkernel}

\newcommand{\svmprob}{\decisionfunction_{svm}}
\newcommand{\svmprobcoeffA}{A}
\newcommand{\svmprobcoeffB}{B}
\newcommand{\borderprob}{\decisionfunction_{border}}

%multi-classes:
\newcommand{\nclass}{n_c}
\newcommand{\multiprob}{p}


\tableofcontents

\section*{List of symbols}

%list of symbols as a latex table


\begin{tabular}{lll}
symbol & definition & first used \\ \hline
	$\lindecision$ & decision function in linear classifier & (\ref{linear_classifier})\\
	$\bordernormal$ & hyperplane normal of decision boundary & (\ref{linear_classifier})\\
	$\borderconst$ & constant value defining location of decision boundary & (\ref{linear_classifier})\\
	$\testpoint$ & test point & (\ref{linear_classifier})\\
	$\densityestimator$ & kernel density estimator & (\ref{kernel_estimator})\\
	$\vectorkernel$ & kernel function taking two vector arguments & (\ref{kernel_estimator})\\
	$\lbrace \sample_i \rbrace$ & set of training samples & (\ref{kernel_estimator})\\
	$\kernelparam$ & parameters in vector kernel & (\ref{kernel_estimator})\\
	$\norm$ & kernel normalization coefficient & (\ref{kernel_estimator})\\
	$\nsample$ & number of training samples & (\ref{kernel_estimator})\\
	$\point=\lbrace \coord_i \rbrace$ & point in the feature space & (\ref{norm_def})\\
	$\scalarkernel$ & kernel function taking a single scalar argument & (\ref{scalar_kernel_def})\\
	$\bandwidth$ & ``bandwidth'': sets the size of a scalar kernel & (\ref{scalar_kernel_def})\\
	$\probability$ & true probability density & (\ref{setting_the_bandwidth})\\
	$\dimension$ & number of dimensions in feature space & (\ref{setting_the_bandwidth})\\
	$\classlabel_i$ & class label of $i$th training sample & (\ref{kernel_classification})\\
	$\class$ & class label & (\ref{kernel_classification})\\
	$\condprob$ & conditional probability & (\ref{rdef})\\
	$\kernelsum$ & sum of the kernels at each training sample & (\ref{agf_def})\\
	$\svmcoeff$ & coefficients in weighted kernel estimator & (\ref{weighted_kernel_estimator})\\
	$\expandedspace$ & theoretical expanded feature space in kernel-based SVM & (\ref{phi_def})\\
	$\svmdecision$ & raw decision function in binary SVM & (\ref{decision_function0})\\
	$\svmcost$ & cost parameter in SVM for reducing over-fitting & (\ref{svmcost}) \\
	$\diffcondprob$ & difference in conditional probabilities & (\ref{rdef}) \\
	$\decisionfunction$ & decision function which is estimator of $\diffcondprob$ & (\ref{rdef})\\
	$\kerneldecision$ & kernel estimator for $\diffcondprob$ & (\ref{kernel_decision}) \\
	$\vbdecision$ & variablekernel estimator for $\diffcondprob$ & (\ref{vb_kernel_r}) \\
	$\lbrace \bordervector_i \rbrace$ & set of vectors defining the class border & (\ref{border_decision}) \\
	$\lbrace \bordernormal_i \rbrace$ & set of normals to the class border & (\ref{border_decision}) \\
	$\nborder$ & number of border vectors & (\ref{border_decision}) \\
	$\borderdecision$ & raw decision function for border classification & (\ref{border_decision}) \\
	$\distance$ & distance in feature space & (\ref{kernel_gradient}) \\
	$\scalarkernelderiv$ & derivative of a scalar kernel, $\scalarkernel$ & (\ref{kernel_gradient}) \\
	$\svmprob$ & LIBSVM estimator for $\diffcondprob$ & (\ref{svm_prob}) \\
	$\svmprobcoeffA$ & coefficient used in $\svmprob$ & (\ref{svm_prob}) \\
	$\svmprobcoeffB$ & coefficient used in $\svmprob$ & (\ref{svm_prob}) \\
	$\borderprob$ & borders estimator for $\diffcondprob$ & (\ref{border_probability}) \\
	$\lmult$ & Lagrange multiplier in multi-class problem & (\ref{multiclass}) \\
	$\rbkernelparam$ & parameter used for Gaussian kernels in SVM & S \ref{results_section} \\
	$\lbrace \confusion_{ij} \rbrace$ & confusion matrix & (\ref{accuracy}) \\
	$\ntest$ & number of test points & (\ref{accuracy}) \\
	$\priorentropy$ & entropy of the prior distribution & (\ref{prior_entropy}) \\
	$\posteriorentropy$ & entropy of the posterior distribution & (\ref{posterior_entropy}) \\
	$\UC$ & uncertainty coefficient (normalized channel capacity) & (\ref{uncertainty_coefficient})
\end{tabular}





\section{Theory}

\subsection{Kernel estimation}

A kernel is a scalar function of two vectors. Kernel functions are used for 
non-parametric density estimation. A typical ``kernel-density estimator"
looks like this:
\begin{equation}
	\densityestimator(\testpoint) = \frac {1}{\norm \nsample} \sum_{i=1}^\nsample \vectorkernel(\sample_i, \testpoint, \kernelparam)
\end{equation}
where $\densityestimator$ is an estimator for the density, $\probability$, 
$\vectorkernel$ is the kernel function,
$\lbrace \sample_i \rbrace$ are a set of samples, 
$\nsample$ is the number of samples,
$\testpoint$ is the test point,
and $\kernelparam$ is a set of parameters. 
The normalization coefficient, $\norm$, normalizes $\vectorkernel$:
\begin{equation}
	\norm = \int_V \vectorkernel(\point, \point^\prime, \kernelparam) \mathrm d \point^\prime
\end{equation}

The method can be used for statistical classification by simply comparing
results from the different classes:
\begin{equation}
	\class = \arg \min_i \sum_{j|\classlabel_j = i} K(\sample_j, \testpoint, \kernelparam)
\end{equation}
where $\classlabel_j$ is the class of the $j$th sample. Naturally the method can also
be used to estimate both the joint and conditional probabilities 
($\condprob(\class, \vec x)$ and $\condprob(\class |\vec x)$ respectively). 
More on this later.

If the same kernel is used for every sample and every test point, the estimator
may be sub-optimal, particularly in regions of very high or very low density.
\citep{Terrell_Scott1992, Mills2011}.
There are at least two solutions to this problem.
In a ``variable-bandwidth'' estimator, the coefficients, $\kernelparam$, depend in some
way on the density. Since the density itself is normally unavailable, the
estimated density is used as a proxy. In \citet{Terrell_Scott1992} and
\citet{Mills2011} 
Let the kernel function takes the following form:
\begin{equation}
	\vectorkernel(\point, \point^\prime, \bandwidth) = \scalarkernel \left (\frac{|\point - \point^\prime|}{\bandwidth} \right )
\end{equation}
where $\bandwidth$ is the ``bandwidth''. 
In \citet{Mills2011}, $\bandwidth$ is made proportional to the density:
\begin{equation}
	\bandwidth \propto \frac{1}{\probability^{1/\dimension}} \approx \frac{1}{\densityestimator^{1/\dimension}}
\end{equation}
where $\dimension$ is the dimension of the feature space.
Since the normalization coefficient, $\norm$, must include the factor,
${\bandwidth^\dimension}$, some rearrangement shows that:
\begin{equation}
	\frac{1}{\nsample} \sum_i \scalarkernel \left (\frac{|\testpoint - \sample_i|}{\bandwidth} \right ) = \kernelaverage = const.
\end{equation}
This is a generalization of the $k$-nearest-neighbours scheme in which the
free parameter, $\kernelaverage$, takes the place of $k$. \citep{Mills2009, Mills2011}
The bandwidth, $h$, can be solved
for using any numerical, one-dimensional root-finding algorithm.
The bandwidth is determined at each test point.

Another method of improving the performance of a kernel-density estimator
is to add a series of variable coefficients:
\begin{equation}
	\densityestimator(\testpoint) = \sum_i \svmcoeff_i \vectorkernel(\sample_i, \testpoint, \kernelparam)
\end{equation}
The coefficients, $\lbrace \svmcoeff_i \rbrace$, are found through an optimization
procedure designed to minimize the error [citation...]. In the most popular
form of this kernel method, Support Vector Machines (SVM), the coefficients
are the result of a complex, dual optimization procedure which minimizes
the classification error. We will briefly outline this procedure.

\subsection{Support Vector Machines}

The basic ``trick'' of kernel-base SVM methods is to replace a dot product
with the kernel function in the assumption that it can be rewritten
as a dot product of a transformed and expanded feature space:
\begin{equation}
	\vectorkernel(\point, \point^\prime) = \expandedspace(\point) \cdot \expandedspace(\point^\prime)
\end{equation}
For simplicity we have ommitted the kernel parameters.
$\expandedspace$ is a vector function of the feature space.
The simplest example of a kernel function that has a closed, analytical and
finite-dimensional $\expandedspace$ is the square of the dot product:
\begin{eqnarray}
	\vectorkernel(\point, \point^\prime) & = & (\point \cdot \point^\prime)^2 \\ \nonumber
					 & = & (\coord_1^2, \coord_2^2, \coord_3^2, ..., \sqrt{2} \coord_1 \coord_2, \sqrt{2} \coord_1 \coord_3, ... \sqrt{2} \coord_2 \coord_3, ...) \cdot \\ \nonumber
      & &	 ({\coord^\prime_1}^2, {\coord^\prime}_2^2, {\coord^\prime}_3^2, ..., \sqrt{2} \coord^\prime_1 \coord^\prime_2, \sqrt{2} \coord^\prime_1 \coord^\prime_3, ... \sqrt{2} \coord^\prime_2 \coord^\prime_3, ...) 
\end{eqnarray}
but it should be noted that in more complex cases, 
there is no need to construct $\expandedspace$ since it is replaced by the 
kernel function, $\vectorkernel$, in the final analysis.

In a binary SVM classifier, the classes are separated by a single hyper-plane
defined by $\svmbordervector$ and $\svmborderconst$.
In a kernel-based SVM, this hyperplane bisects not the regular feature
space, but the theoretical, transformed space defined by the function,
$\expandedspace(\point)$.
The {\it decision value} is calculated via a dot product:
\begin{equation}
	\svmdecision(\testpoint)=\svmbordervector \cdot \expandedspace(\testpoint) + \svmborderconst
	\label{decision_function0}
\end{equation}
and the class determined by the sign of the decision value:
\begin{equation}
	\class=\frac{\svmdecision}{|\svmdecision|}
	\label{class_value}
\end{equation}
where for convenience, the class labels are given by $c \in \lbrace -1, 1 \rbrace$.

In the first step of the minimization procedure, 
the magnitude of the border normal, $\svmbordervector$, 
is minimized subject to the constraint that there are no classification 
errors:
\begin{eqnarray}
	\min_{\svmbordervector, \svmborderconst} \frac{1}{2} | \svmbordervector | \\ \nonumber
	\svmdecision(\sample_i) \classlabel_i \ge 1
\end{eqnarray}
Introducing the coefficients, $\lbrace \svmcoeff_i \rbrace$, 
as Lagrange multipliers on the constraints:
\begin{equation}
	\min_{\svmbordervector, b} \left \lbrace \frac{1}{2} | \svmbordervector | - \sum_i \svmcoeff_i \left [ \svmdecision(\sample_i) \classlabel_i -1 \right ] \right \rbrace
\end{equation}
generates the following pair of analytic expressions:
\begin{eqnarray}
	\sum_i \svmcoeff_i \classlabel_i & = & 0 \\
	\svmbordervector & = & \sum_i \svmcoeff_i \classlabel \expandedspace(\sample_i) \label{border_vector_equation}
\end{eqnarray}
through setting the derivatives w.r.t. the minimizers to zero.
Substituting the second equation, (\ref{border_vector_equation}),
into the decision function in (\ref{decision_function0}) produces the following:
\begin{equation}
	\svmdecision(\testpoint) = \sum_i \svmcoeff_i \classlabel_i \vectorkernel (\testpoint, \sample_i) + \svmborderconst
	\label{svm_decision}
\end{equation}
Meanwhile, the final, dual, quadratic optimization problem looks like this:
\begin{eqnarray}
	\max_{\lbrace \svmcoeff_i \rbrace} & \sum \svmcoeff_i 
	- \frac{1}{2} \sum_{i, j} \svmcoeff_i \svmcoeff_j \classlabel_i \classlabel_j \vectorkernel(\sample_i, \sample_j) \\ \nonumber
	& \svmcoeff_i \ge 0 \\ \nonumber
	& \sum_i \svmcoeff \classlabel_i = 0 \label{dual_problem}
\end{eqnarray}
There are a number of refinements that can be applied to the optimization
problem in (\ref{dual_problem}), chiefly to reduce over-fitting and to add
some ``margin'' to the decision border to allow for the possibility of
classification errors, but mainly we are concerned with the decision
function in (\ref{svm_decision}) since the initial fitting will be done with
an external software package, namely LIBSVM \citep{Chang_Lin2011}.

Two things should be noted. First, the function $\expandedspace$ appears in
neither the final decision function, (\ref{svm_decision}), nor in the
optimization problem, (\ref{dual_problem}). Second, while the use of
$\expandedspace$ implies that the time complexity of the decision function 
could be $O(1)$ as in a parametric statistical model, in actual fact it is
dependent on the number of non-zero values in $\lbrace \svmcoeff \rbrace$.
While the coefficient set, $\lbrace \svmcoeff \rbrace$, does tend to be sparse,
nonetheless in most real problems the number of non-zero coefficients is
proportional to the number of samples, $\nsample$, producing a time complexity
of $O(\nsample)$. Thus for large problems, calculating the decision value will
be slow, just as in other kernel estimation problems.

The advantage of SVM lies chiefly in its accuracy since it is minimizing the
classification error whereas a more basic kernel method is more ad hoc
and does little more than sum the number of samples weighted by distance.

\subsection{Border classification}

\label{border_method}

In kernel SVM, the decision border exists only implicitly in a hypothetical,
abstract space. Even in linear SVM, if the software is generalized to 
recognize the simple dot product as only one among many possible kernels,
then the decision function may be built up, as in (\ref{svm_decision})
through a sum of weighted kernels. This is the case for LIBSVM.
The advantage of an explicit decision border as in (\ref{decision_function0})
is that it is fast. The problem with a linear border is that, except for a
small class of problems, it is not very accurate.

In the binary classification method described in \citet{Mills2011},
a non-linear decision border is built up piece-wise from a collection of linear borders.
It is essentially a root-finding procedure for a decision function,
such as $\svmdecision$ in (\ref{svm_decision}).
Let $\decisionfunction$ be a decision function. In the ideal case, it should
be an accurate estimator for the difference in conditional probabilities:
\begin{equation}
	\decisionfunction(\point) = \diffcondprob(\point) = 
	\condprob(1|\point) - \condprob(-1|\point)
\end{equation}
where $\condprob(\class|\point)$ represents the conditional probabilities of
a binary classifier having labels $\class \in \lbrace -1, 1 \rbrace$.
For a simple kernel estimator, for instance, 
$\diffcondprob$ is estimated as follows:
\begin{equation}
	\kerneldecision(\testpoint) = \frac{\sum_i \classlabel_i \vectorkernel(\testpoint, \sample_i)}{\sum_i \vectorkernel(\testpoint, \sample_i)}
	\label{kernel_decision}
\end{equation}
where $\classlabel_i \in \lbrace -1, 1 \rbrace$.
For a variable bandwidth kernel estimator, this works out to:
\begin{equation}
	\vbdecision(\testpoint) = \frac{1}{\kernelaverage} \sum_i \classlabel_i \scalarkernel \left (\frac{|\testpoint - \sample_i|}{\bandwidth(\testpoint)} \right )
	\label{vb_kernel_r}
\end{equation}
A variable bandwidth kernel-density estimator with a Gaussian kernel,
$\scalarkernel(x)=\exp(-x^2/2)$ we will refer to as an ``Adaptive Gaussian
Filter'' or AGF for short.

The procedure is as follows: pick a pair of points on either side of the decision
boundary (the decision function has opposite signs). Good candidates are one
random training sample from each class. Then, zero the decision function
along the line between the points. This can be done as many times as needed
to build up a good representation of the decision boundary.
We now have a set of points, $\lbrace \bordervector_i \rbrace$, such that
$\decisionfunction(\bordervector_i)=0$ for every $i=[1..\nborder]$ where
$\nborder$ is the number of border samples.

Along with the border samples,  $\lbrace \bordervector_i \rbrace$, we also
collect a series of normal vectors, $\lbrace \bordernormal_i \rbrace$
such that:
\begin{equation}
\bordernormal_i=\nabla_{\point}{\decisionfunction |_{\point=\bordervector_i}}
\end{equation}
With this system, determining the class is a two step proces.
First, the nearest border to the test point is found.
Second, we define a new decision function, $\borderdecision$, 
similar to (\ref{decision_function0}), through a dot product with the normal:
\begin{eqnarray}
	i & = & \arg \min_j |\testpoint - \bordervector_j| \\ \nonumber
	\borderdecision(\testpoint) & = & \bordernormal_i \cdot (\testpoint - \bordervector_i)
	\label{border_decision}
\end{eqnarray}
The class is determined by the sign of the decision function as in 
(\ref{class_value}).
The time complexity is completely independent of the number
of training samples, rather it is linearly proportional to the number of
border vectors, $\nborder$, a tunable parameter. The number required for
accurate classifications is dependent on the complexity of the decision
border.

The gradient of the variable-bandwidth kernel estimator in 
(\ref{vb_kernel_r}) is:
\begin{equation}
	\frac{\partial \vbdecision}{\partial \testcoord_j} = 
	\frac{2 \bandwidth}{\kernelaverage} \sum_i \classlabel_i
	\scalarkernelderiv \left (\frac{\distance_i}{\bandwidth} \right )
	\left [\frac{\testcoord_j - \samplecoord_{ij}}{\distance_i} 
	- \distance_i \frac{\sum_k \scalarkernelderiv \left (\frac{\distance_k}{\bandwidth} \right )
	\frac{\testcoord_j - \samplecoord_{kj}}{\distance_k}}
{\sum_k \distance_k \scalarkernelderiv \left (\frac{\distance_k}{\bandwidth} \right )} \right ]
\end{equation}
where $\distance_i=|\point - \sample_i|$ is the distance between the 
test point and the $i$th sample and $\scalarkernelderiv$ is the derivative
of $\scalarkernel$.
For AGF, this works out to:
\begin{equation}
	\frac{\partial \vbdecision}{\partial \testcoord_j} = 
	\frac{1}{\bandwidth^2 \kernelaverage} \sum_i \classlabel_i
	\scalarkernel \left (\frac{\distance_i}{\bandwidth} \right )
	\left [\samplecoord_{ij} - \testcoord
	- \distance_i^2 \frac{\sum_k \scalarkernel \left (\frac{\distance_k}{\bandwidth} \right )
	\left (\samplecoord_{kj} - \testcoord \right )}
{\sum_k \scalarkernel \left (\frac{\distance_k}{\bandwidth} \right )\frac{1}{\distance_k^2}} \right ]
\end{equation}
where $K(x)=\exp(-x^2/2)$ \citep{Mills2011}.

In LIBSVM, conditional probabilities are estimated by applying a
sigmoid function to
the raw SVM decision function, $\svmdecision$, in (\ref{svm_decision}):
\begin{equation}
	\svmprob(\testpoint) = \tanh \left (\frac{\svmprobcoeffA \svmdecision(\testpoint)+ \svmprobcoeffB}{2} \right )
	\label{svm_prob}
\end{equation}
where $\svmprobcoeffA$ and $\svmprobcoeffB$ are coefficients derived from
the training data via
nonlinear least-squares fitting \citep{Lin_etal2007, Chang_Lin2011}.

The gradient of the revised SVM decision function, above, is:
\begin{equation}
	\nabla_{\point} {\svmprob} = \left [1 - \svmprob^2(\point) \right ] \sum_i \svmcoeff_i \classlabel_i \nabla_{\point} \vectorkernel(\point, \sample_i)
\end{equation}

Gradients of the initial decision function are useful not just to derive normals to
the decision boundary, but also as an aid to root finding when searching for
border samples. If the decision function used to compute the border samples
represents an estimator for the
difference in conditional probabilities, then the raw decision value,
$\borderdecision$,
derived from the border sampling technique in (\ref{border_decision})
can also return estimates of the conditional probabilities with little
extra effort and little loss of accuracy--also using a sigmoid function:
\begin{equation}
	\borderprob(\testpoint) = \tanh \left [\borderdecision (\testpoint) \right ]
	\label{border_probability}
\end{equation}
This assumes that the class posterior probabilities
($\condprob(\point | c)$) are approximately Gaussian near the border
\citep{Mills2011}.

The border classification algorithm, returns an estimator,
$\borderprob$, for the difference in conditional probabilities of
a binary classifier using
equations (\ref{border_decision}) and (\ref{border_probability}).
It can be trained with the functions $\kerneldecision$ in (\ref{kernel_decision}),
$\vbdecision$ in (\ref{vb_kernel_r}), $\svmprob$ in (\ref{svm_prob}),
or any other 
continuous, differentiable, non-parametric estimator for the difference
in conditional probabilities, $\diffcondprob$.
At the cost of a small reduction in accuracy,
it has the potential to drastically reduce classification time for kernel
estimators and other non-parametric statistical classifiers,
especially for large training datasets,
since it has $O(\nborder)$ time complexity instead of $O(\nsample)$
complexity, where $\nborder$, the number of border samples, is a free parameter.
The actual number chosen
can trade off between speed and accuracy with rapidly diminishing returns
beyond a certain point. 
One hundred border samples ($\nborder=100$) is usually sufficient.
The computation of $\borderprob$ also involves very simple operations--
floating point addition, multiplication and numerical comparison, with no
transcendental functions except for the very last step (which can be omitted)--so the coefficient for the time complexity will be small.

A border classifier trained with AGF will be referred to as an ``AGF-borders''
classifier while a border classifier trained with SVM estimates will
be referred to as an ``SVM-borders'' classifier.

\subsection{Multi-class classification}

The border classification algorithm, like SVM, only works for binary 
classification problems. It is quite easy to generalize a binary classifier
to perform multi-class classifications by using several of them and the
number of ways of doing so grows exponentially with the number of classes.
Since LIBSVM uses the ``one-versus-the-rest'' method \citep{Hsu_Lin2002} of 
multi-class classification, this is the one we will adopt. 

A major advantage of the
borders classifier is that it returns probability estimates.
These estimates are based on a simple interpolation, however, so they may
not always be accurate and we are interested in comparing their accuracy
with that of the original estimates. Therefore, the multi-class method
should also solve for the conditional probabilities in addition to returning
the class label.

In a one-vs.-the rest scheme, the multi-class conditional probabilities 
can be related to those of the binary classifiers as follows:
\begin{equation}
	\diffcondprob_{ij}(\point) = \frac{\condprob(j|\point) - \condprob(i|\point)}{\condprob(i|\point) + \condprob(j|\point)}
\end{equation}
where $i=[1..\nclass-1]$, $\nclass$ is the number of classes, $j>i$,
and $\diffcondprob_{ij}$ is the difference in conditional probabilities of
the binary classifier that discriminates between the $i$th and $j$th classes.
\citet{Wu_etal2004} transform this problem into the following linear system:
\begin{eqnarray}
	\sum_i \frac{1}{4} \left [ \multiprob_i \sum_{j|j \ne i} (\decisionfunction_{ij} + 1)^2 - \multiprob_i (\decisionfunction_{ij}^2 - \decisionfunction_{ij} - 2) \right ] + \lmult & = & 0 \\
	\sum_i \multiprob_i & = & 1
\end{eqnarray}
where $\multiprob_i = \condprob(i | \point)$ is the $i$th multi-class 
conditional probability and $\lmult$ is a Lagrange multiplier.
They also show that it can be solved iteratively so that the constraints 
not included in the problem, that the probabilities are all positive, are satisfied.

\section{A simple example}

\begin{figure}
\includegraphics[width=0.9\textwidth]{support_vectors}
\label{sample_sv}
\end{figure}

\begin{figure}
\includegraphics[width=0.9\textwidth]{border_vectors}
\label{sample_border}
\end{figure}

\begin{figure}
\includegraphics[width=0.9\textwidth]{skill_v_ns}
\label{skill_v_ns}
\end{figure}

We use the pair of synthetic test classes from \citet{Mills2011} to illustrate
the difference between support vectors and border vectors and between border
vectors derived from AGF and from a LIBSVM model.
Figure \ref{sample_sv} shows a realization of the two sample classes 
in red and blue, comprising 300 samples total, along
with the support vectors derived from a LIBSVM model.
The support vectors are a subset of the training samples and while they
tend to cluster around the border, they do not define it.
For reference, the border between the two classes is also shown.
This has been derived from the border classification method described in 
Section \ref{border_method} using the mathematical definition of the classes,
hence it represents the ``true'' border to within a numerical error.

The true border is also compared with those derived from AGF and LIBSVM
probability estimates in Figure \ref{sample_border}.
The classes are once again shown for reference.
While these borders contain 1000 samples for a clear view of where
they are located using each method, in fact the method works well with
surprisingly few samples.  Figure \ref{skill_v_ns} shows a plot of the skill
as a function of the number of border samples, where {\it U.C.} stands for
uncertainty coefficient. Note that the scores saturate at only about 20
samples meaning that for this problem at least, very fast classifications are
possible.

\bibliography{../agf_bib}

\end{document}
