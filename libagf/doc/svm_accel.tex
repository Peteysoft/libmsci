\documentclass{article}

\title{When Support Vector Machines are just too slow}

\section{Preliminaries}

\subsection{Kernel estimation}

A kernel is a scalar function of two vectors. Kernel functions are used for 
non-parametric density estimation. A typical ``kernel-density estimator"
looks like this:
\begin{equation}
	f(\vec x) = \frac {1}{N n} \sum_i K(\vec x_i, \vec x, \vec k)
\end{equation}
where $f$ is the density estimator, $K$ is the kernel function,
$\lbrace \vec x_i \rbrace$ are a set of samples, $n$ is the number of samples,
$\vec x$ is the test point,
and $\vec k$ is a set of coefficients. 
The normalization coefficient, $N$, normalizes $K$:
\begin{equation}
	N = \int_V K(\vec x, \vec x^\prime, \vec k) \mathrm d \vec x^\prime
\end{equation}

If the same kernel is used for every sample and every test point, the estimator
may be sub-optimal, particularly in regions of very high or very low density.
\citep{Terrell_Scott1992, Mills2011}.
There are at least two solutions to this problem.
In a ``variable-bandwidth'' estimator, the coefficients, $\vec k$, depend in some
way on the density. Since the density itself is normally unavailable, the
estimated density is used as a proxy. In \citet{Terrell_Scott1992} and
\citet{Mills2011} 
Let the kernel function takes the following form:
\begin{equation}
	K(\vec x, \vec y, h) = g \left (\frac{|\vec x - \vec y|}{h} \right )
\end{equation}
where $h$ is the ``bandwidth''. In \citet{Mills2011}, $h$ is made
proportional to the density:
\begin{equation}
	h \propto \frac{1}{P^D} \approx \frac{1}{f^D}
\end{equation}
Since the normalization coefficient, $N$, must include the factor,
${1}{h^D}$, some rearrangement shows that:
\begin{equation}
	\sum_i g \left (\frac{|\vec x - \vec x_i|}{h} \right ) = W = const.
\end{equation}
This is a generalization of the $k$-nearest-neighbours scheme in which the
free parameter, $W$, takes the place of $k$. The bandwidth, $h$, can be solved
for using any numerical, one-dimensional root-finding algorithm.

Another method of improving the performance of a kernel-density estimator
is to add a series of variable coefficients:
\begin{equation}
	f(\vec x) = \sum_i c_i K(\vec x_i, \vec x, \vec k)
\end{equation}
