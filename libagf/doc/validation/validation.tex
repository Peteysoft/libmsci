\documentclass{article}

\usepackage{natbib}

\bibliographystyle{apa}

\newcommand{\probability}{P}
\newcommand{\condprob}{P}
\newcommand{\estimate}{\tilde p}
\newcommand{\coord}{x}
\newcommand{\features}{\vec \coord}

\begin{document}

\title{A simple method of validating classification conditional probabilities}

\author{Peter Mills}

\maketitle

\section{Introduction}

In regression and statistical classification, estimates may be validated
through direct comparision with the training data.
With density estimation, the existance of training data does not provide
a means of direct validation.
Here we are interested in validating conditional probability estimates from a
statistical classification algorithm.
\citet{Mills2009} describes a simple method of validation based on Bayes
theorem and the definition of probability.
The goal of this note is to validate this validation method.

\section{Theory}

From Bayes theorem and the definition of probability, we can write:
\begin{equation}
	\int P(\vec x) P(c | \vec x) \mathrm d \vec x = \lim_{N \rightarrow \infty} N_c/N
\end{equation}
where $N_c$ is the number of training samples of class $c$ and $N$ is the 
total number of training samples.
Transforming this into a Monte Carlo sum (with importance sampling):
\begin{equation}
	\lim_{N \rightarrow \infty} \left [ \sum_{i=1}^N P(c | \vec x_i) = N_c \right ]
\end{equation}
where $\vec x_i$ is the features data of $i$th training sample.
We use the Kronecker delta 
to move the equals sign outside the limit:
\begin{equation}
	\lim_{N \rightarrow \infty} \left [ \sum_{i=1}^N P(c | \vec x_i) - \delta_{cy_i} \right ] = 0 \label{theorem}
\end{equation}
where $y_i$ is the class of $i$th training sample.

To make the theorem useful for validation purposes there are a few things 
to be done.
First, the estimated conditional probabilities, $\tilde p$, are sorted from
low to high:
\begin{equation}
	\tilde p(c_{i+1}|\vec x_{i+1}) \ge \tilde p(c_i|\vec x_i)
\end{equation}
Note that there is no need to separate out each class, hence the subscript
on the class label.
Second, we define the index, $i_0$, as the rank of the first ``winning''
probability estimate:
\begin{equation}
	\tilde p(c_i | \vec x_i) \ge 1/n \iff i \ge i_0
\end{equation}
where $n$ is the number of classes.
With this in place, Equation (\ref{theorem}) is rearranged 
and divided into twp parts.
The first part counts up:
\begin{equation}
	N^\prime \approx 
	\sum_{i=i_0}^{N^\prime+i_0} \frac{\delta_{c_iy_i}}{\tilde p(c_i | \vec x_i)} \\
\end{equation}
while the second part counts down:
\begin{equation}
	N^\prime \approx \sum_{i=N^\prime+i_0}^{i_0} \frac{\delta_{c_iy_i} - 1}{1 - \tilde p(c_i | \vec x_i)}
\end{equation}

With this arrangement, the trace should follow the diagonal on average and
move at single step intervals.
Dividing it up prevents singularities in the denominator while picking the
threshold of $1/n$ means that most of the terms in both expressions should
be non-zero.


\section{Derivation}

Suppose the probability event $c$ for a given trial is constant at $p_0$.
There are $N$ trials each with value $y_i$. 
If $y_i=c$ then the event has occured, otherwise not.
The number of times the event occurs divided by the total number of
measurements will approach the probability as more measurements are added:
\begin{equation}
p_0 = \lim_{N \rightarrow \infty} \frac{1}{N}\sum_{i=1}^N \delta_{cy_i}
\end{equation}
where $\delta$ is the Kronecker delta.

The same principle will hold true for a classification condtional probability,
$P(c|\vec x)$, except that the conditional probability will be able to take
on multiple values, so we exclude all those for which $P(c|\vec x) \ne p_0$:
\begin{equation}
p_0 = \lim_{N_0 \rightarrow \infty} \frac{1}{N_0} \sum_{P(c|\vec x_i)=p_0} \delta_{cy_i}
\end{equation}
where $\lbrace \vec x_i : y_i | i \in [1, N] \rbrace$ is a set of classification
training data 
and $N_0$ is the number of training samples for which $P(c|\vec x)=p_0$,
$|\lbrace i | P(c|\vec x_i) = p_0 \rbrace |$.
We rearrange this as follows:
\begin{eqnarray}
\lim_{N_0 \rightarrow \infty} \left [\sum_{P(c|\vec x_i)=p_0} \delta_{cy_i} - N_0 p_0 \right ] & = & 0 \\
\lim_{N_0 \rightarrow \infty} \sum_{P(c|\vec x_i)=p_0} \left [\delta_{cy_i} - p(c | \vec x_i) \right ] & = & 0 \label{constant_prob}
\end{eqnarray}

We could imagine summing a whole set of equations in the form of 
(\ref{constant_prob}) with different constant probabilities, $p_0, p_1, p_2, ...$:
\begin{equation}
\sum_j \lim_{N_j \rightarrow \infty} \sum_{P(c|\vec x_i)=p_j} \left [\delta_{cy_i} - p(c | \vec x_i) \right ] = 0
\end{equation}
\begin{equation}
\lim_{N_0 \rightarrow \infty} \lim_{N_1 \rightarrow \infty} \lim_{N_2 \rightarrow \infty} ... \sum_i \left [ P(c|\vec x_i) - \delta_{cy_i} \right ] = 0
\end{equation}
Some rearrangment produces the following theorem:
\begin{equation}
\lim_{N \rightarrow \infty} \sum_i \left [ P(c|\vec x_i) - \delta_{cy_i} \right ] = 0
\end{equation}
where $N=\sum_j N_j$.

\bibliography{../agf_bib}

\end{document}

