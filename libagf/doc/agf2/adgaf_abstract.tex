Adaptive Gaussian filtering (AGF) is a new method for estimating
probability densities and performing statistical classification.
Its justification is a generalization of a $k$-nearest-neighbours
whereby the training samples are weighted by distance
using a symmetric filter function and the total of the weights 
is constrained to a constant value by varying the width of the
filter.  A Gaussian is chosen 
because of its mathematical properties--it is easy to solve
for the filter width, for instance.  
Probability estimates 
can be used directly or they can be used to search for a 
class border from which estimates of both the
class and the conditional probabilities are easy to extrapolate.
The method is validated on a pair of two-dimensional synthetic
test classes and compared
to three of the most popular existing methods.  
As applied to the test case it is shown
to be over twenty-five (25) times as fast as the equivalent
analysis using a support vector machine (SVM), with no loss of
accuracy, while being easier to apply and understand.
