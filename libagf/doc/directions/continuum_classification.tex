%\documentclass[12pt]{article}

%\usepackage{amsmath}

%\begin{document}

\section{Applying AGF to continuum retrievals}

Here we discuss methods for applying statistical classification towards
estimation of continuous quantities.

\subsection{Conditional probabilities as a proxy}

If the statistical classification method returns conditional probabilities,
these can sometimes serve as a good proxy for the continuous quantity.
First, assume that the statistics are Gaussian and write the conditional
probability of the continuous variable:
\begin{equation}
P(y|\vec x) = \frac{1}{\sqrt {2 \pi} \sigma} 
	\left \lbrace \frac{[\bar y(\vec x) - y]^2}{2 \sigma^2} \right \rbrace
\end{equation}
where $\bar y$ is the mean as a function of position in the feature-space,
$\vec x$ and $\sigma$ is the dispersion.

Suppose there are only two classes, with the class division at $y_0$.
Define $R(\vec x)$ as the difference in conditional probabilities:
\begin{eqnarray}
R & = & P(2 | \vec x) - P(1 | \vec x) \\
& = & \int_{y_0}^\infty P(y | \vec x) - \int_{-\infty}^{y_0} P(y|\vec x) \\
& = & \mathrm{erf} \left [ \frac{\bar y(\vec x) - y_0}{\sqrt 2 \sigma} \right ]
\end{eqnarray}
Thus the difference in conditional probabilities is related to the error
function of the mean, $\bar y$, divided by the standard deviation, $\sigma$.

In ``borders classification'', the class of a test point is found by finding
the nearest of a set of border samples and taking the dot product of the
gradient with the displacement vector:
\begin{eqnarray}
j & = & \arg \min_i |\vec b_i - \vec x| \\
p & = & (\vec x - \vec b_j) \cdot \nabla_{\vec x} R |_{\vec x=\vec b_j} 
\label{pdef} \\
c & = & (3+p/|p|)/2
\end{eqnarray}
where $\vec x$ is the test point and $\lbrace \vec b_i \rbrace$ are the border
samples.  
In previous analysis, $R$ was approximated by taking the hyperbolic tangent of
the result at (\ref{pdef}), however from the preceding, it's apparent that the
error function, which is also sigmoid like $\tanh$, 
would be more appropriate:
\begin{equation}
R \approx \mathrm{erf}(p)
\end{equation}
The approximation holds so long as
$\bar y$ is linear in $\vec x$:
\begin{equation}
\bar y(\vec x) \approx y_0 + \sqrt 2 \sigma (\vec x - \vec b_j) \cdot \nabla_{\vec x} R |_{\vec x=\vec b_j}
\end{equation}
If there are multiple class borders, or iso-surfaces, $y_0$, this is not so
hard to enforce as follows below.

This development assumes that the dispersion, $\sigma$, is known,
perhaps through experiment.
We can, however, easily sample two or more iso-surfaces, 
$\lbrace y_0^{(i)} \rbrace$
within the feature space, $\vec x$.  For a given test point, we have:
\begin{equation}
p_i(\vec x) = \frac{\bar y(\vec x) - y_0^{(i)}}{\sqrt 2 \sigma}
\label{baryandp}
\end{equation}
Using two iso-surfaces, 
$i$ and $j$---presumably the closest---we can solve for both
$\sigma$ and $\bar y$:
\begin{eqnarray}
\bar y & = & \frac{p_i y_0^{(j)} - p_j y_0^{(i)}}{p_i - p_j} \\
\sigma & = & \frac{\bar y - y_0^{(i)}}{\sqrt 2 p_i}
\end{eqnarray}
If the surfaces bracket the test point, then it is a simple, linear
interpolation, using $|p|$ as the distance measure. 
Note that in this case, $p_i$ and $p_j$ have opposite sign.
If the surfaces do not
bracket the test point
then $p_i$ and $p_j$ have the same sign and estimates tend to
blow up.

In most cases we will have multiple realizations of $p$.  We propose the
following least-squares method, weighted by $p$:
\begin{eqnarray}
a_1 & = & - \sqrt 2 \sum_{i=1}^n \frac{1}{p_i} \\
a_2 & = & \sum_{i=1}^n \frac{1}{p_i^2} \\
b_1 & = & - \sqrt 2 \sum_{i=1}^n \frac{y_0^{(i)}}{p_i} \\
b_2 & = & \sum_{i=1}^n \frac{y_0^{(i)}}{p_i^2} \\
d & = & 2 n a_2 - a_1^2 
\end{eqnarray}
\begin{eqnarray}
\bar y & = & (2 n b_2 - a_1 b_1)/d \\
\sigma & = & (a_2 b_1 - a_1 b_2)/d
\end{eqnarray}
Where $n$ is the number of iso-surfaces.  This can be derived by applying the normal equation to (\ref{baryandp}).
While this scheme appears to suffer from two flaws---first, $\bar y$ is being
linearly extrapolated no matter how far the iso-surface, and second,
$\sigma$ is constant---however each item is weighted inversely with the 
value of $p$ which is a measure of distance from the iso-surface.

%\end{document}

