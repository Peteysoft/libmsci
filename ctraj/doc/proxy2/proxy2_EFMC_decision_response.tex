\documentclass{letter}

\usepackage{url}

\newcommand{\vect}[1]{\vec{#1}}

\signature{Peter Mills}

\address{Peter Mills\\
1159 Meadowlane\\
Cumberland ON\\
K4C 1C3\\
1-613-833-3022\\
\url{mailto://peteymills@hotmail.com}
}

\begin{document}

\begin{letter}{Stefano Galmarini\\
Journal of Environmental Fluid Mechanics}

\opening{Dear Dr. Galmarini, journal editors and referees:}

I have incorporated the changes recommended by the referees to the manuscript
entitled, ``PC proxy: a method of dynamical tracer reconstruction,'' currently
submitted to the {\it Journal of Environmental Fluid Mechanics}. 
I have also made other changes to improve the clarity, style and accuracy.
Point-by-point responses are given below.

\begin{quote}
COMMENTS TO THE AUTHOR:

Reviewer \#1: General comment:

The manuscript focuses on the development and validation of a diagnostic tool for stratospheric ozone concentration, based on PCA (Principal Component Analysis).

Provided the results reported on the manuscript and the major issues below, the reviewer suggests a major revision.

Major issues:

1.On the method formulation and implementation.

1.1.The few main features of the method need much more details and explanations. For example, Sec.2.2 needs a major revision to provide a clearer explanation of its main contents. Some of the following issues provide some further details.

1.2.Please provide the main details on the planetary-scale air quality model used for the tracer transport/advection, included the vertical extension of the numerical domain.
\end{quote}
All the numerical trials in this work were performed in only two dimensions.
It would be straightforward, however, to extend the method to three dimensions.
\begin{quote}
Please report the main balance equations of the transport/dispersion model.
\end{quote}
The transport model is described in detail in Section 3.1, ``Tracer simulation.''
\begin{quote}
Please provide a detailed description of the proxy tracer simulation, the numerical tracer source, the dimensions of its plume.
\end{quote}
There are no sources because all of the tracer simulations applied in this
work are completely passive. There is no plume, because except for the example
in Section 4.1, no actual tracers are explicitly advected.
\begin{quote}
1.3.In PCA, one reduces ``n'' variables into ``m\textless n'' PCs (Principal Components) to reduce the computational time (but losing some accuracy). The presented method seems to reduce ``m'' measures of the stratospheric ozone concentration (one variable) to ``k'' PCs. Please clarify in detail this issue.
\end{quote}
The method reduces the transport matrix, which is a representation of the passive
tracer dynamics, to $k$ principal components which are then correlated with the
measurements.
Neither the size of the transport matrix nor the number of measurements is
mentioned explicity since the math can be presented perfectly well without 
either parameter.
It should be clear from the exposition that the transport matrix must be square.
\begin{quote}
1.4.The correlation of the stratospheric ozone concentration with a simulated tracer concentration (in the stratosphere) needs detailed explanations and proofs. No reference (on relevant journals) is provided on the state-of-the-art behind this major hypothesis.
\end{quote}
Please check reference [1]. The correlation between passive or long-lived tracers over time is well documented. In fact we even provide a numerical demonstration in Section 4.1. The phenomenon has been used to estimate stratospheric ozone as in reference [13] (Randall et al; [14] in the revision).
\begin{quote}
1.5.Eq(3). A minus sign seems missed at the beginning of the LHS (Left Hand Side) -also in (5) and (7)-.
\end{quote}
Good catch. This has been fixed.
\begin{quote}
Please highlight that this refers to a 1D formulation, no boundary condition is considered and no stability criterion is provided.
\end{quote}
``As an illustration, consider a second-order finite difference scheme in one dimension...''
Since it is just an example used for illustrative purposes only, I see no need to mention the boundary conditions. 
To clarify further, I have added the following: ``The actual transport model used in this study is described in Section 3.1.''
\begin{quote}
1.6.Eq.(17) and Eq.(19). Please provide details on the estimation of the interpolation coefficients ``w\_i''.
\end{quote}
``In this work, bilinear interpolation is used to interpolate measurement
locations.'' has been added.
\begin{quote}
1.7.Sec.2.3. No relevant reference is provided on the method itself.
\end{quote}
We provide a total of four references for the ``classic'' proxy tracer method.
I would suggest that reference [1] by Allen and Nakamura, is the most
relevant.
\begin{quote}
1.8. Eq.(20). This equation seems incompatible with Eq.(19). For example, ``c'' depends on ``i'' instead of ``j''; ``k'' is not defined.
\end{quote}
We have added a ``prime'' symbol to $c$ to distinguish it from the $c$ used in
PC proxy and replaced $k$ with $i$ which is then defined just below the equation. Thanks for spotting the error in Equation (20).
\begin{quote}
1.9.Eq(27). Please verify if a time scale is missing.
\end{quote}
Thanks. This has been fixed.
\begin{quote}
1.10.Eq.(45). The variable ``s'' should be explicitly defined.
\end{quote}
``where $s$ is the path'' has been added.
\begin{quote}
1.11.Sec.4.1. Please specify where the tracer sources are located and the correlation coefficient is estimated.
\end{quote}
``Tracers are passively advected...''
``The Pearson coefficient, weighted by grid size, is applied over the whole field at a single time-step.''
\begin{quote}
1.12.Sec.5. This method might work locally as a diagnostic tool coupled with a numerical model for the tracer. Please discuss the advantages and shortcomings of using such a modeling system instead of a planetary-scale air quality model with 4D data assimilation.
\end{quote}
``Principal component proxy tracer reconstruction is shown to be a powerful
technique that 
has many of the advantages of prognostic assimilation models but with
less of the associated complexity.'' I have also added:
``In particular, there is no need for explicitly modelled sources and sinks."
\begin{quote}
2.On the method validation.

2.1.Please highlight the possible risks of having too few input measuring points, which might be responsible for providing extrapolated instead of interpolated results in many parts of the investigated stratospheric level. The distribution of the sondes might be too irregular and sparse to provide validation for a diagnostic model. Errors might grow uncontrolled due to extrapolations replacing interpolation. Please discuss this issue.
\end{quote}
This issue is discussed in some detail in Section 6, third and fourth
paragraphs.
Since the interpolated tracer fields are not themselves integrated and there are no feedback loops, the error
cannot grow uncrontrollably.
\begin{quote}
2.2. For the comparisons (correlation plots), error statistics should be reported either on all the figures or collected in a unique table. In any case, they should be normalized. Please estimate fractional bias, Normalized Mean Square Error, FAC2 and hit rate.
\end{quote}
A table has been added summarizing the results.
Please note that some of the values have changed because in calculating the
new skill scores, I have re-run all the experiments. I have doubtless refined
the software since the previous run.
\begin{quote}
2.3.Page 14. Final period. ``In the histogram error plot in Figure 7b. the errors have been normalized by the original error estimates for the POAM III retrievals.'' This procedure seems not adequate enough: please avoid this correction or properly justify it.
\end{quote}
I have added the following sentence which explains the reasoning:
``This makes it easy to compare residuals with the estimated errors
for the original ozone estimates."
\begin{quote}
2.4.Sec.5.1. Please provide the positions of the two ensembles of POAM grid points (the one to estimate the interpolation parameters, the other to compare with the results) and their average spatial resolution.
\end{quote}
A figure has been added which plots all of the POAM measurement locations used
in the study.

\begin{quote}
Minor issues:
1. Abstract. Please specify that this study refers to the spatial resolution of the planetary-scale.
\end{quote}
``Using a 60 day integration time and five (5) principal components, cross
validation of globally reconstructed ozone...''
\begin{quote}
2. Abstract.  ``long-lived tracers such as ozone'' should be replaced with ``long-lived atmospheric pollutants such as ozone'' or equivalent expression.
\end{quote}
The phrase no longer appears in the abstract.
\begin{quote}
3. Abstract. ``with ozone measurements'' should be replaced with ``with stratospheric ozone concentration measurements'' or equivalent expression.
\end{quote}
OK
\begin{quote}
4.Sec.2.``D is the diffusivity tensor.'' might be replaced with ``D is the dispersion tensor'' or equivalent expression.
\end{quote}
I have replaced it with ``diffusion tensor''
\begin{quote}
5.``The linear operator'' should be replaced with ``The non-linear operator''.
6.Eqs.(3)-(4). No need for 2 equation IDs. Please revise the numbering of all the equations since.
\end{quote}
I have included two proofs of linearity at the end of this letter. Both are too long, unfortunately, to include in the manuscript. Numbering for all multi-line equations in
the main body has been revised.
\begin{quote}
7.Sec.2.2. The symbol ``m\_i'' might be misleading. Please consider introducing ``q\_m\_i'' to be coherent with ``q''.
\end{quote}
This is not a bad idea and was done in an earlier version of the manuscript but could cause confusion with the other two uses of the symbol, $q$.
\begin{quote}
8.Sec.2.4. ``the tangent vector is simply the dynamics''. Please revise this expression.
\end{quote}
I have removed this phrase. The meaning, in any case, should be clear from the derivation.
\begin{quote}
9.Sec.2.6. ``The error is simply the terms ... wind field.'' All the error contributions should be taken into account. Here, only the method error in the continuum is considered. Please refer to the "method error in the continuum" or equivalent expression.
\end{quote}
I have changed this sentence to read, ``The most significant source of error are the terms left out of the equation...''
\begin{quote}
10.The software ``ctraj'' needs a proper citation among the references.
\end{quote}
I have cited two earlier works, both of which used an earlier version of the same software. The first one in particular provides a very detailed description of the algorithms.
\begin{quote}
11.Fig.3. Please verify whether the minima relate to the square of the correlation coefficient or not.
\end{quote}
I don't understand this statement. The figure plots the correlation coefficient.
\begin{quote}
12.Sec.4.2. Third to last line. Please revise the equality.
\end{quote}
Do you mean ``derive''? A derivation has been included in the appendix.
\begin{quote}
13.Figures. ``ozone'' should be replaced with ``stratospheric ozone concentration''.
\end{quote}
``Stratospheric'' is implicit in the 500 K isentrope which is stated in 
every figure caption. The units of ozone concentration are clearly marked on the axis labels.
\begin{quote}
14.``correlation'' might be replaced with ``correlation coefficient'' (all over the text).
15.Sec.5.2. ``of 0.64, a bias'' should be replaced with ``of 0.51, a bias''.
16.Fig.12. Caption. Please report ``Arctic POAM III only'' or an equivalent expression.
\end{quote}
OK.

\begin{quote}
Reviewer \#2: This is a high quality research and extremely well written paper. Congratulations to the author for this work. I enjoyed reading it. I think the method the author is proposing for reconstruction of tracer transport field is promising and worth of publication. From the technical standpoint I have no major comments as I think the paper is clear enough to allow replication and widespread use. My biggest and only concern with this paper is that it seems that the topic was already published in a previous paper by the author (see ``Principal component proxy tracer analysis.'' arXiv preprint arXiv:1202.1999 (2012)). arXiv (where this paper seems published)  is a repository of scientific works that as far as I know doesn't undergo the scrutiny of the peer review process. These are electronic prints of work but I am not sure if there are copyright restrictions to these published works. I recommend the author to explain if this research was already published in arXiv and if
so, to indicate what are the additions or further contributions that make this paper an original and new contribution for consideration in the Journal of Environmental Fluid Mechanics.

If the manuscript available in arXiv is not considered a publication, my suggestion is to accept this manuscript in its present form.
\end{quote}

Thank you very much. This paper is a revised and expanded version (based on reviewer comments) of the earlier preprint.

I look forward to your response.

\closing{Kind regards}

{\Large Proof of linearity 1}

\begin{equation}
\frac{\partial q(\vect x, ~ t)}{\partial t} = \left \lbrace -\vect v(\vect x, ~t) \cdot \nabla + \nabla \cdot D \nabla \right \rbrace q(\vect x, ~ t)
\label{advection_diffusion}
\end{equation}


%\section{Proof of linearity}

\label{linearity}

%(in case you skipped first-year mathematics...)

A linear operator has the following two properties:
\begin{eqnarray}
	L(ka) & = & kLa \\
	L(a+b) & = & La + Lb
\end{eqnarray}
where $L$ is a linear operator and $k$ is a constant.
We test the operator in (\ref{advection_diffusion}) for each of these two
properties by writing it out component-by-component:
\begin{eqnarray}
	\frac{\partial q}{\partial t} & = & \left \lbrace -\vect v \cdot \nabla + \nabla \cdot D \nabla \right \rbrace q \\
				   & = & - \sum_i v_i \frac{\partial q}{\partial x_i} + \sum_i \frac{\partial}{\partial x_i} \left ( \sum_j d_{ij} \frac{\partial q}{\partial x_j} \right )
\end{eqnarray}
We test the first property by substituting $kq$ for $q$:
\begin{eqnarray}
	\left \lbrace -\vect v \cdot \nabla + \nabla \cdot D \nabla \right \rbrace (kq) & = & - \sum_i v_i \frac{\partial kq}{\partial x_i} + \sum_i \frac{\partial}{\partial x_i} \sum_j d_{ij} \frac{\partial kq}{\partial x_j} \\
	& = & - \sum_i v_i k \frac{\partial q}{\partial x_i} + \sum_i \frac{\partial}{\partial x_i} \sum_j k d_{ij} \frac{\partial q}{\partial x_j} \\
	& = & - \sum_i k v_i \frac{\partial q}{\partial x_i} + \sum_i \frac{\partial}{\partial x_i} \left ( k \sum_j d_{ij} \frac{\partial q}{\partial x_j} \right ) \\
	& = & - k \sum_i v_i \frac{\partial q}{\partial x_i} + k \sum_i \frac{\partial}{\partial x_i} \sum_j d_{ij} \frac{\partial q}{\partial x_j} \\
	& = & k \left ( - \sum_i v_i \frac{\partial q}{\partial x_i} + \sum_i \frac{\partial}{\partial x_i} \sum_j d_{ij} \frac{\partial q}{\partial x_j} \right ) \\
	& = & k \left \lbrace -\vect v \cdot \nabla + \nabla \cdot D \nabla \right \rbrace q
\end{eqnarray}
We test the second property by substituting $q_1 + q_2$ for $q$:
\begin{eqnarray}
	& & \left \lbrace -\vect v \cdot \nabla + \nabla \cdot D \nabla \right \rbrace (q_1 + q_2) \\
	& = & - \sum_i v_i \frac{\partial (q_1+q_2)}{\partial x_i} + \sum_i \frac{\partial}{\partial x_i} \sum_j d_{ij} \frac{\partial (q_1+q_2)}{\partial x_j} \\
	& = & - \sum_i v_i \left ( \frac{\partial q_1}{\partial x_i} + \frac{\partial q_2}{\partial x_i} \right ) + \sum_i \frac{\partial}{\partial x_i} \sum_j d_{ij} \left ( \frac{\partial q}{\partial x_j} + \frac{\partial q_2}{\partial x_j} \right ) \\
%	& = & \sum_i \left ( v_i \frac{\partial q_1}{\partial x_i} + v_i \frac{\partial q_2}{\partial x_i} \right ) + \sum_i \frac{\partial}{\partial x_i} \left ( \sum_j d_{ij} \frac{\partial q}{\partial x_j} + \sum_j d_{ij} \frac{\partial q_2}{\partial x_j} \right ) \\
	& = & - \left (\sum_i v_i \frac{\partial q_1}{\partial x_i} + \sum_i v_i \frac{\partial q_2}{\partial x_i} \right ) + \sum_i \frac{\partial}{\partial x_i} \left ( \sum_j d_{ij} \frac{\partial q_1}{\partial x_j} + \sum_j d_{ij} \frac{\partial q_2}{\partial x_j} \right ) \\
	& = & - \sum_i v_i \frac{\partial q_1}{\partial x_i} + \sum_i \frac{\partial}{\partial x_i} \sum_j d_{ij} \frac{\partial q_1}{\partial x_j} - \sum_i v_i \frac{\partial q_2}{\partial x_i} + \sum_i \frac{\partial}{\partial x_i} \sum_j d_{ij} \frac{\partial q_2}{\partial x_j} \\
	& = & \left \lbrace -\vect v \cdot \nabla + \nabla \cdot D \nabla \right \rbrace q_1 + \left \lbrace -\vect v \cdot \nabla + \nabla \cdot D \nabla \right \rbrace q_2
\end{eqnarray}




{\Large Proof of linearity 2}


%\section{Proof of linearity}

\label{linearity}

%(in case you skipped first-year mathematics...)

A linear operator has the following two properties:
\begin{eqnarray}
	L(ka) & = & kLa \\
	L(a+b) & = & La + Lb
\end{eqnarray}
where $L$ is a linear operator and $k$ is a constant.

We can test the operator in (\ref{advection_diffusion}) for each of these two
properties by writing it out component-by-component:
\begin{eqnarray}
	\frac{\partial q}{\partial t} & = & \left \lbrace -\vect v \cdot \nabla + \nabla \cdot D \nabla \right \rbrace q \\
				   & = & - \sum_i v_i \frac{\partial q}{\partial x_i} + \sum_i \frac{\partial}{\partial x_i} \left ( \sum_j d_{ij} \frac{\partial q}{\partial x_j} \right )
\end{eqnarray}
This is straightforward, however here we aim for a more axiomatic approach.

First we show that the product of linear operators is a linear operator:
\begin{eqnarray}
	L_1 L_2 (ka) & = & L_1 k L_2 a \\
		    & = & k L_1 L_2 a
\end{eqnarray}
\begin{eqnarray}
	L_1 L_2 (a+b) & = & L_1 (L_2 a + L_2 b) \\
		    & = & L_1 L_2 a + L_1 L_2 b
\end{eqnarray}

The sum of linear operators is a linear operator:
\begin{eqnarray}
	(L_1 + L_2) (ka) & = & L_1 (ka) + L_2 (ka) \\
		    & = & k L_1 a + k L_2 a \\
	     & = & k (L_1 a + L_2 a) \\
	     & = & k (L_1 + L_2) a
\end{eqnarray}
\begin{eqnarray}
	(L_1 + L_2) (a + b) & = & L_1 (a + b) + L_2 (a + b) \\
		    & = & L_1 a + L_1 b + L_2 a + L_2 b \\
	     & = & (L_1 a + L_2 a) + (L_1 b + L_2 b) \\
	     & = & (L_1 + L_2) a + (L_1 + L_2) b
\end{eqnarray}

The gradient ($\nabla$) operator is linear:
\begin{eqnarray}
\nabla (k \vec r) & = & \left (\frac{\partial}{\partial x_1}, ~
	\frac{\partial}{\partial x_2}, ~ \frac{\partial}{\partial x_3}
	\right ) \left [k (r_1, ~ r_2, ~ r_3) \right ] \\
	& = & \left [\frac{\partial}{\partial x_1} (k r_1), 
	~ \frac{\partial}{\partial x_2} (k r_2), 
	~ \frac{\partial}{\partial x_3}	(k r_3) \right ] \\
	& = & \left (k \frac{\partial r_1}{\partial x_1} , 
	~ k \frac{\partial r_2}{\partial x_2}, 
	~ k \frac{\partial r_3}{\partial x_3} \right ) \\
& = & k \left (\frac{\partial}{\partial x_1}, ~
	\frac{\partial}{\partial x_2}, ~ \frac{\partial}{\partial x_3}
	\right ) (r_1, ~ r_2, ~ r_3) \\
	& = & k \nabla \vec r
\end{eqnarray}
\begin{eqnarray}
\nabla (\vec a + \vec b) 
& = & \left [\frac{\partial}{\partial x_1} (a_1 + b_1), ~
\frac{\partial}{\partial x_2} (a_2 + b_2), 
~ \frac{\partial}{\partial x_3} (a_3 + b_3)
	\right ]\\
	& = & \left (\frac{\partial a_1}{\partial x_1}
	+ \frac{\partial b_1}{\partial x_1},
	~ \frac{\partial a_2}{\partial x_2} 
	+ \frac{\partial b_2}{\partial x_2},
	~ \frac{\partial a_3}{\partial x_3} 
	+ \frac{\partial b_3}{\partial x_3} \right ) \\
	& = & \left (\frac{\partial a_1}{\partial x_1},
	~ \frac{\partial a_2}{\partial x_2},
	~ \frac{\partial a_3}{\partial x_3} \right ) + 
	\left ( \frac{\partial b_1}{\partial x_1},
	~ \frac{\partial b_2}{\partial x_2},
	~ \frac{\partial b_3}{\partial x_3} \right ) \\
	& = & \nabla \vec a + \nabla \vec b
\end{eqnarray}

The dot product is a linear operator:
\begin{eqnarray}
	\vec r \cdot (k \vec x) & = & \sum_i r_i (k x_i) \\
			     & = & k \sum_i r_i x_i \\
	       & = & k \vec r \cdot \vec x
\end{eqnarray}
\begin{eqnarray}
	\vec r \cdot (\vec a + \vec b) & = & \sum_i r_i (a_i+b_i) \\
			     & = & \sum_i r_i a_i + \sum_i r_i b_i \\
	       & = & \vec r \cdot \vec a + \vec r \cdot \vec b
\end{eqnarray}

Matrix multiplication is a linear operator:
\begin{eqnarray}
	A (k \vec x) & = & \sum_j a_{ij} (k x_j) \\
	  & = &  k \sum_j a_{ij} x_j \\
	  & = & k A \vec x
\end{eqnarray}
\begin{eqnarray}
	R (\vec a + \vec b) & = & \sum_j r_{ij} (a_j + b_j) \\
		& = & \sum_j r_{ij} a_j + \sum_j r_{ij} b_j \\
	 & = & R \vec a + R \vec b
\end{eqnarray}

Finally, we show that the divergence operator is a linear operator:
\begin{eqnarray}
	\nabla \cdot (k \vec r) & = & \sum_i \frac{\partial}{\partial x_i} (k r_i) \\
	& = & k \sum_i \frac{\partial r_i}{\partial x_i} \\
	& = & k \nabla \cdot \vec r
\end{eqnarray}
\begin{eqnarray}
	\nabla \cdot (\vec a + \vec b) & = & \sum_i \frac{\partial}{\partial x_i} (a_i + b_i) \\
	& = & \sum_i \frac{\partial a_i}{\partial x_i}
	+ \sum_i \frac{\partial b_i}{\partial x_i} \\
	& = & \nabla \cdot \vec a + \nabla \cdot \vec b
\end{eqnarray}

The operator in Equation (\ref{advection_diffusion}) is composed of sums and
products of gradients, dot products, matrix multiplication and divergence
operators. Therefore, it too is a linear operator.



\end{letter}

\end{document}
